\documentclass[12pt]{amsart}
\usepackage{amsmath}
\usepackage{amsthm}
\usepackage{amsfonts}
\usepackage{amssymb}
\usepackage[margin=1in]{geometry}
\usepackage{stackengine}
\usepackage{hyperref}
\hypersetup{
    colorlinks=true,
    linkcolor=blue
}

\theoremstyle{definition}
\newtheorem{theorem}{Theorem}[section]
\newtheorem{lemma}[theorem]{Lemma}
\newtheorem{definition}[theorem]{Definition}
\newtheorem{corollary}[theorem]{Corollary}
\newtheorem{proposition}[theorem]{Proposition}
\newtheorem{conjecture}[theorem]{Conjecture}
\newtheorem{remark}[theorem]{Remark}
\newtheorem{example}[theorem]{Example}
\newtheorem{problem}[theorem]{Problem}
\newtheorem{notation}[theorem]{Notation}
\newtheorem{question}[theorem]{Question}
\newtheorem{caution}[theorem]{Caution}

\begin{document}

\title{Homework}

\maketitle

For this week, please answer the following questions from the text. 
I've copied the problem itself below and the question numbers for 
your convenience. 

\begin{enumerate}
	\item (3.24) For each of the following numbers $N$, compute 
		the values of 
	\begin{displaymath}
		N+1^2,N+2^2,N+3^2,N+4^2,\ldots 
	\end{displaymath}
	as we did in Example 3.34 until you find a value $N+b^2$ that 
	is a perfect square $a^2$. Then the use the values of $a$ and 
	$b$ to factor $N$.
	\begin{enumerate}
		\item $N = 53357$
		\item $N = 34571$
		\item $N = 25777$
		\item $N = 64213$
	\end{enumerate}

	\item (3.26) For each part, use the data provided to find the 
		values of $a$ and $b$ satisfying $a^2 = b^2 \mod n$, and 
		then compute $\operatorname{gcd}(N,a-b)$ in order to 
		find a nontrivial factor of $N$, as we did in Examples 
		3.37 and 3.38.
	\begin{enumerate}
		\item $N = 61063$ 
		\begin{alignat*}{3}
			1882^2 & = 270 \mod 61063~ && \operatorname{and}~
			270 && = 2 \cdot 3^3 \cdot 5 \\
			1898^2 & = 60750 \mod 61063~ && \operatorname{and}~ 
			60750 && = 2 \cdot 3^5 \cdot 5^3  \\
		\end{alignat*}
		\item $N = 52907$
		\begin{alignat*}{3}
			399^2 & = 480 \mod 52907~ && \operatorname{and}~
			480 && = 2^5 \cdot 3 \cdot 5 \\
			763^2 & = 192 \mod 52907~ && \operatorname{and}~
			192 && = 2^6 \cdot 3 \\ 
			773^2 & = 15552 \mod 52907~ && \operatorname{and}~
			15552 && = 2^6 \cdot 3^5 \\
			976^2 & = 250 \mod 52907~ && \operatorname{and}~
			250 && = 2 \cdot 5^3 
		\end{alignat*}
		\item $N = 198103$
		\begin{alignat*}{3}
			1189^2 & = 27000 \mod 198103~ && \operatorname{and}~
			27000 && = 2^3 \cdot 3^3 \cdot 5^3 \\
			1605^2 & = 686 \mod 198103~ && \operatorname{and}~
			686 && = 2 \cdot 7 \\ 
			2378^2 & = 108000 \mod 198103~ && \operatorname{and}~
			108000 && = 2^5 \cdot 3^3 \cdot 5^3 \\
			2815^2 & = 105 \mod 198103~ && \operatorname{and}~
			105 && = 3 \cdot 5 \cdot 7
		\end{alignat*}
		\item $N = 2525891$
		\begin{alignat*}{3}
			1591^2 & = 5390 \mod 2525891~ && \operatorname{and}~
			5390 && = 2 \cdot 5 \cdot 7^2 \cdot 11 \\
			3182^2 & = 21560 \mod 2525891~ && \operatorname{and}~
			21560 && = 2^3 \cdot 5 \cdot 7^2 \cdot 11 \\ 
			4773^2 & = 108000 \mod 2525891~ && \operatorname{and}~
			108000 && = 2^5 \cdot 3^3 \cdot 5^3 \\
			2815^2 & = 105 \mod 2525891~ && \operatorname{and}~
			105 && = 3 \cdot 5 \cdot 7
		\end{alignat*}
			
	\end{enumerate}
		

		\begin{enumerate}
			\item $N = 61063$ 
			\begin{alignat*}{3}
				1882^2 & = 270 \mod 61063~ && \operatorname{and}~
				270 && = 2 \cdot 3^3 \cdot 5 \\
				1898^2 & = 60750 \mod 61063~ && \operatorname{and}~ 
				60750 && = 2 \cdot 3^5 \cdot 5^3  \\
			\end{alignat*}
			\item $N = 52907$
			\begin{alignat*}{3}
				399^2 & = 480 \mod 52907~ && \operatorname{and}~
				480 && = 2^5 \cdot 3 \cdot 5 \\
				763^2 & = 192 \mod 52907~ && \operatorname{and}~
				192 && = 2^6 \cdot 3 \\ 
				773^2 & = 15552 \mod 52907~ && \operatorname{and}~
				15552 && = 2^6 \cdot 3^5 \\
				976^2 & = 250 \mod 52907~ && \operatorname{and}~
				250 && = 2 \cdot 5^3 
			\end{alignat*}
			\item $N = 198103$
			\begin{alignat*}{3}
				1189^2 & = 27000 \mod 198103~ && \operatorname{and}~
				27000 && = 2^3 \cdot 3^3 \cdot 5^3 \\
				1605^2 & = 686 \mod 198103~ && \operatorname{and}~
				686 && = 2 \cdot 7^3 \\ 
				2378^2 & = 108000 \mod 198103~ && \operatorname{and}~
				108000 && = 2^5 \cdot 3^3 \cdot 5^3 \\
				2815^2 & = 105 \mod 198103~ && \operatorname{and}~
				105 && = 3 \cdot 5 \cdot 7
			\end{alignat*}
			\item $N = 2525891$
			\begin{alignat*}{3}
				1591^2 & = 5390 \mod 2525891~ && \operatorname{and}~
				5390 && = 2 \cdot 5 \cdot 7^2 \cdot 11 \\
				3182^2 & = 21560 \mod 2525891~ && \operatorname{and}~
				21560 && = 2^3 \cdot 5 \cdot 7^2 \cdot 11 \\ 
				4773^2 & = 48510 \mod 2525891~ && \operatorname{and}~
				48510 && = 2 \cdot 3^2 \cdot 5 \cdot 7^2 \cdot 11 \\
				5275^2 & = 40824 \mod 2525891~ && \operatorname{and}~
				40824 && = 2^3 \cdot 3^6 \cdot 7 \\
				5401^2 & = 1386000 \mod 2525891~ && \operatorname{and}~
				1386000 && = 2^4 \cdot 3^2 \cdot 5^3 \cdot 7 \cdot 11 \\
			\end{alignat*}
		\end{enumerate}
	\item (3.27) Compute the followimg values of $\psi(X,B)$, the number of $B$-smooth 
		numbers between $2$ and $X$ (see page 150).
		\begin{enumerate}
			\item $\psi(25,3)$
			\item $\psi(35,5)$
			\item $\psi(50,7)$
			\item $\psi(100,5)$
			\item $\psi(100,7)$
		\end{enumerate}

	\item (3.34) Illustrate the quadratic sieve, as was down in Fig. 3.3 (page 161), 
		by sieving prime powers up to $B$ on the values of $F(T) = T^2 - N$ in 
		the indicated range.
		\begin{enumerate}
			\item Sieve $N=493$ using prime powers up to $B=11$ on values 
				from $F(23)$ to $F(38)$. Use the relation(s) that you 
				find to factor $N$. 
			\item Extend the computation in (a) by using prime powers up 
				to $B=16$ and sieving values from $F(23)$ to $F(50)$. 
				What additional value(s) are sieved down to $1$ and 
				what additional relation(n) do they yield? 
		\end{enumerate}
			
	\item (3.35) Let $\mathbb{Z}[\beta]$ be the ring described in Example 3.55, i.e. 
		$\beta$ is a root of $f(x) = 1 + 3x - 2x^3 + x^4$. For each of the 
		following pairs of elements $u,v \in \mathbb{Z}[\beta]$, compute 
		the sum $u + v$ and the product $uv$. Your answers should only involve 
		powers of $\beta$ up to $\beta^3$. 
		\begin{enumerate}
			\item $u = -5-2\beta+9\beta^2-9\beta^3$ and $v=2+9\beta-7
				\beta + 7 \beta^2$.
			\item $u = 9 + 9\beta + 6\beta^2 - 5\beta^3$ and $v = -4 
				- 6\beta - 2\beta^2 -5 \beta^3$.
			\item $u = 6 - 5\beta + 3\beta^2 + 3 \beta^3$ and $v = -2 
				+ 7\beta + 6\beta^2$. 
		\end{enumerate}
			
\end{enumerate}
\end{document}
