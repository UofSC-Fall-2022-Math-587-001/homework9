\documentclass[12pt]{amsart}
\usepackage{amsmath}
\usepackage{amsthm}
\usepackage{amsfonts}
\usepackage{amssymb}
\usepackage[margin=1in]{geometry}
\usepackage{stackengine}
\usepackage{hyperref}
\hypersetup{
    colorlinks=true,
    linkcolor=blue
}

\theoremstyle{definition}
\newtheorem{theorem}{Theorem}[section]
\newtheorem{lemma}[theorem]{Lemma}
\newtheorem{definition}[theorem]{Definition}
\newtheorem{corollary}[theorem]{Corollary}
\newtheorem{proposition}[theorem]{Proposition}
\newtheorem{conjecture}[theorem]{Conjecture}
\newtheorem{remark}[theorem]{Remark}
\newtheorem{example}[theorem]{Example}
\newtheorem{problem}[theorem]{Problem}
\newtheorem{notation}[theorem]{Notation}
\newtheorem{question}[theorem]{Question}
\newtheorem{caution}[theorem]{Caution}

\begin{document}

\title{Homework}

\maketitle

For this week, please answer the following questions from the text. 
I've copied the problem itself below and the question numbers for 
your convenience. 

\begin{enumerate}
	\item (3.17) The function $\pi(X)$ counts the number of primes between $2$ and $X$. 
		\begin{enumerate}
		\item Compue the values of $\pi(20),\pi(30)$, and $\pi(100)$. 
		\item Write a program to compute $\pi(X)$ and use it to compute $\pi(X)$ and 
			the ratio $\pi(X)/(X/\ln X)$ for $X = 100, 1000, 10000,$ and $100000$. 
			Does your list of ratios make the prime number theorem plausible? 
		\end{enumerate}
	\item (3.19) We noted in Sect. 3.4 that it really makes no sense to say
		that the number n has probability $1/\ln n$ of being prime. Any
		particular number that you choose either will be prime or will
		not be prime; there are no numbers that are 35 \% prime and 65 \%
		composite! In this exercise you will prove a result that gives
		a more sensible meaning to the statement that a number has a
		certain probability of being prime. You may use the prime
		number theorem (Theorem 3.21) for this problem.
		\begin{enumerate}
			\item Fix a (large) number N and suppose that Bob
			chooses a random number $n$ in the interval $N/2
			\leq n \leq 3N/2$ . If he repeats this process many
			times, prove that approximately $1/\ln N$ of his
			numbers will be prime. More precisely, define
			\begin{align*}
				P(N) & := \frac{\text{number of primes between $N/2$ and 
					$3N/2$}}{\text{number of integers between $N/2$ 
					and $3N/2$}} \\
				     & = \left[\Centerstack{\text{Probability that a random integer $n$ 
			     in the interval} \newline \text{$N/2 \leq n \leq 3N/2$ is a prime number}} \right]
			\end{align*}
			and prove that 
			\begin{displaymath}
				\lim_{N \to \infty} \frac{P(N)}{1/\ln N} = 1
			\end{displaymath}
			\item More generally, fix two numbers $c_1$ and $c_2$ satisfying $c_1 > 
				c_2 > 0$. Bob chooses random numbers $n$ in the interval 
				$c_1 N \leq n \leq c_2 N$. Keeping $c_1$ and $c_2$ fixed, let 
			\begin{displaymath}
				P(c_1,c_2;N) := \left[\Centerstack{\text{Probablity that an integer 
					$n$ in the interval} \newline \text{$c_1N \leq n \leq 
					c_2 N$ is a prime number}}\right]
			\end{displaymath}
			In the following formula, fill in the gox with a simple function of $N$ 
			so that the statement is true. 
			\begin{displaymath}
				\lim_{N \to \infty} \frac{P(c_1,c_2;N)}{\framebox(3em,1.5em){}} = 1. 
			\end{displaymath}
		\end{enumerate}
	\item (3.23) A prime of the form $2^n-1$ is called a \textit{Mersenne prime}. 
		\begin{enumerate}
			\item Factor each of the numbers $2^n-1$ for $n=2,3,\ldots,10$. Which ones 
				are Mersenne primes?
			\item Find the first seven Mersenne primes. (You may need a computer.)
			\item If $n$ is even and $n > 2$, prove that $2^n-1$ is not prime.
			\item If $3 \mid n$ and $n > 3$, prove that $2^n-1$ is not prime.
			\item More generally, prove that if $n$ is a composite number, then 
				$2^n-1$ is not prime. Thus all Mersenne primes have the form 
				$2^p-1$ with $p$ a prime number. 
			\item What is the largest known Mersenne prime? Are there any larger 
				primes known? (You can find out at the "Great Internet Mersenne 
				Prime Search" web site \url{www.mersenne.org/prime.htm}.)
		\end{enumerate}
			
			
\end{enumerate}
\end{document}
